\documentclass{article}
\usepackage{graphicx} % Required for inserting images
\usepackage{hyperref}
\title{\huge {\textbf{Level-2}}}
\author{FPGAspeaks}
\date{August 2023}
\usepackage{geometry}
\geometry{a4paper, margin=2cm}
\begin{document}

\maketitle

\section{Introduction}
In this level you will learn System Design in Verilog. This level is divided into two parts namely
\textbf{Level-2.A} and \textbf{Level-2.B}. In this level Test-Benches are also provided.
\section{Pre-requisites}
Basic knowledge of Digital System Design. Even if you don't know just learn that particular concept and get started.
\section{If Tools are}
\begin{enumerate}
    \item \textbf{Linux Ubuntu terminal }: As my laptop also slow downs while using such heavy software, I too have done using terminal. Thus, I have created this Level using terminal.\\
    Visit my youtube video \href{https://youtu.be/92rmMcu5vGQ}{Click Here}
     \item \textbf{Xilinx Vivado/Xilinx ISE }: If you are using this tools then you have to remove two lines given below in the testbenches provided
     \begin{enumerate}
         \item \$dumpfile("user-defined-filename.vcd");
         \item \$dumpvars(0,testbenchfilename);
     \end{enumerate}
     Also remove \$finish in the last\\
     or else simply watch my video \href{https://www.youtube.com/watch?v=-aOc2moMT_I}{Click Here} 
\end{enumerate}
\section{What you are supposed to do ? }
\textbf{Level-2.A} is of basic standard and \textbf{Level-2.B} is of medium standard of System Design. In both of this you will find a Order-to-follow.txt file check-out to learn in the sequence. 
\subsection{\underline{Note}} Once you are done with this level make Documentation of what you have learnt and upload in your Git-Hub repository in this particular level. 
\end{document}
